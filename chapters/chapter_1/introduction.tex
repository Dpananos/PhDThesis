\chapter{Introduction}

\section{Motivation}

Different patients can react to a drug differently just by virtue of being different people, even when said patients match on clinical variables.  This between-patient variability in drug response is an obstacle to optimal treatment since some patients may experience heightened response (possibly leading toxicity) while others may experience lowered response (possibly leading to weakened response or inefficacy).  Personalized medicine is a response to this variation, with the goals of: 1) identifying drugs for which between-patient variation is a key issue for effective treatment,  2) to address/identify factors driving this variation, 3) to treat the right patient with the right drug  of the right amount at the right time, 4) and to aid in the prevention of adverse events associated with said drugs \cite{morse2015personalized}.  This thesis is primarily concerned with goals of identifying which factors are drivers of between patient variability in service of better selecting the right dose for the right patient.

To this end, this thesis focuses on pharmacokinetic modelling for both prediction and inference, as well as using pharmacokinetic models for  decision making (be it a single decision or sequences of decisions).