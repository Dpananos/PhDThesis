\chapter{Introduction}

\section{Motivation}

Different patients can react to a drug differently just by virtue of being different people, even when said patients match on clinical variables.  This between-patient variability in drug response is an obstacle to optimal treatment since some patients may experience heightened response (possibly leading to toxicity) while others may experience lowered response (possibly leading to inefficacy).  Personalized medicine is a response to this variation.  Morse and Kim identify four goals for personalized medicine: 1) identify drugs for which between-patient variation is a key issue for effective treatment,  2) address/identify factors driving this variation, 3) treat the right patient with the right drug  of the right amount at the right time, 4) and aid in the prevention of adverse events associated with said drugs \cite{morse2015personalized}.  This thesis is primarily concerned with goals of identifying which factors are drivers of between patient variability in service of better selecting the right dose for the right patient, and is thus aligned with goals 2 and 3.

The setting of interest for this thesis is pharmacokinetics, which has been described by Rosenbaum as concerning itself with the time course of drug concentrations in the body \cite{ rosenbaum2016basic}.
Effective drug response requires adequate systematic exposure.  Since drug concentration is a means of measuring drug exposure,  an understanding which factors drive variation in concentration can give insight into factors driving variation in response.  To this end, many pharmacokinetic modelling studies seek to identify clinical, genetic, and lifestyle factors which are associated with changes in concentration. Some studies will provide dose adjustment criteria based on the results of their modelling. These recommendations are personalized in so far as they identify sub-populations of patients which may see predictably larger/smaller concentrations as a function of standard doses. However, there exists additional concentration variability beyond that observed in clinical trials for some drugs \cite{sukumar2019apixaban}, raising questions about optimal dosing for these drugs.  The discovery of this excess variation in applied settings motivates the ``fine tuning'' or ``tailoring'' of the pharmacokinetic modelling done in previous studies for application in the population of interest.

There may be prohibitive obstacles for conducting new pharmacokinetic studies of the type performed by drug companies in order to tailor predictions to a given population. New studies on pharmacokinetics for drugs can leverage existing knowledge and results through the adoption of a Bayesian framework.  In the Bayesian framework, a prior distribution of plausible parameter values is specified before seeing data.  Once data is collected then the model can be ``updated'' with the data.  The prior distribution allows researchers to ostensibly start with an existing model they think appropriately describes their population of interest, and sequentially update or ``fine tune'' the model as additional data is collected. This workflow of starting with an initial belief about the patient, collecting data, and updating said belief is roughly reflective of how dose titration for drugs like warfarin are performed.   While the Bayesian framework formalizes the process of updating beliefs about pharmacokinetics when observing data from the titration process, the framework of dynamic treatment regimes formalizes the optimal sequential process of selecting the doses for titration.  Chakraborty and Murphy describe dynamic treatment regimes as ``one way to operationalize a clinical decision support system'' \cite{chakraborty_dynamic_2014}, that is to say dynamic treatment regimes are a formalization of the sequential decisions made about treating patients.  Since personalized medicine can be conceived as a set of sequential decisions (on how best to treat a patient based on observations made from a titration process), dynamic treatment regimes offer an enticing set of tools for personalized medicine and see continued use in the field.  Because decisions in personalized medicine can be made on the basis of concentrations (be they too high or too low to elicit a desired response), the incorporation of pharmacokinetic models into dynamic treatment regimes for purposes of sequentially optimal dose titration offers a synergy of two frameworks important to personalized medicine.

This thesis outlines methods for creating Bayesian pharmacokinetic models for two purposes: inference into the effects of covariates on pharmacokinetic parameters and thereby concentrations, as well as use for these models in sequential decisions on dose size.    Bayesian statistics is the formalism adopted so as to incorporate and directly build upon prior information from other pharmacokinetic modelling efforts.  These models are an attempt to address questions pertaining to the second goal of personalized medicine, identification of factors driving variation in drug response.  The models will be directly incorporated into a sequential decision making framework thereby addressing questions pertaining to the third goal of personalized medicine, optimal dosing.


% Bayes
% Fine tuning for the population
% leverge existing work

% Sequential decision making
% The goal of titration is optimal deicsion making, but that is implicit, mathematizing the decision making that happens in clincial practice (Susan Murphy)
% If we can combine PK and DTR then great
% Develop and combine three areas.

\section{Objectives}

I address three objectives motivated by the desire to identify sources of between patient variability and account for these in downstream decisions on drug dosing:

\begin{enumerate}[1)]
	\item To compare and contrast existing approaches to fitting Bayesian models with recent advancements in the pursuit of fitting population pharmacokinetic models. 
	
	\item To develop a framework to evaluate the  benefits of collecting additional information on patients against the additional burden posed on the patient to adhere to additional requirements. 
	
	\item To demonstrate how personalized medicine researchers in academic centres can use all data available to them, even if those data do not come from tightly controlled studies, to study effects of clinical variables on pharmacokinetics while also exploring new variables which may explain additional variation.
\end{enumerate}


\section{Research Contributions}

The research contributions of these thesis are as follows:

\begin{enumerate}
	\item  A one compartment pharmacokinetic model with first order elimination written in an open source probabilistic programming language.  In addition to this, I present a specific parameterization of the model, achieved through non-dimensionalization of the differential equation governing the kinetics, which better facilitates sampling using Hamiltonian Monte Carlo as opposed to a naive parameterization.  Lastly, I present a simulation study demonstrating that inferences made using a popular inference method in pharmacokinetic research leads to different and poorer calibrated decisions as compared to inference methods proposed more recently.
	
	\item A unified framework for the development and simulation-based evaluation of personalization based on pharmacokinetic modelling. I demonstrate how the framework can be applied with a case study, considering six modes of personalization, each of which differ in the amount of information collected from  and burden imposed on the patient (in terms of adhering to additional measurement requirements).  The knowledge created by the framework can be integrated into a system level decision-making framework, such as Know4Go \cite{Martin2016}.  
	
	\item A demonstration of how investigators can fit pharmacokinetic models with aim of accurate modelling of pharmacokinetics and exploration of novel variables.  Importantly, this approach demonstrates how to use \textit{all} data available to investigators.  The data I use come from two very different sources:  one from a well controlled clinical study with strict inclusion/exclusion criteria and the other from a personalized medicine clinic in which subjects are observed only once. In particular, I note that when some studies collect multiple observations from the same patient and others collect only a single sample, traditional modelling approaches which concatenate all data violate the assumption of exchangeability. In response, I offer a reparameterization for which the exchangeability assumption is more defensible.  Lastly, I demonstrate how sparsity inducing priors can be applied to new variables in order to explore how those variables may effect pharmacokinetics, and demonstrate the efficacy of the method in a simulation study.  I show that when come from two different types of studies, like the ones mentioned above, then the bias imparted from the sparsity inducing prior can be sufficiently small so as to be acceptable (especially in light of the exploratory nature of the problem). 
\end{enumerate}

% All of the output from all of the papers
% Matchy matchy with the objectives, that's OK
% Paper 1:  What are the main results and outputs.  Non-dimensionalized models that converge well with modern sampling software. WHat are the new things that exist after the papers that didn't exist before.
% Somewhat more detailed account of the moe specific things.

\section{Thesis Organization}

This thesis is written with an integrated-article format.  \textbf{Chapter 2} provides necessary background on  concepts and terms that are needed to understand the body of the work.  \textbf{Chapter 3} provides a literature review of recent advances in the fields of Bayesian statistics, sequential decision making, and pharmacokinetics, providing context for the three articles to follow.  \textbf{Chapters 4, 5, and 6} include the integrated articles which address objective 1 --- 3 respectively.  \textbf{Chapter 7} concludes the thesis with an overarching discussion and examines possible subsequent areas of investigation.  Any \textbf{appendices} are included at the end of each integrated article, and may contain supporting materials such as extra information on methods, additional exposition for models, and additional visualizations.  Due to the nature of the integrated article format, there is some repetition between introductory sessions.