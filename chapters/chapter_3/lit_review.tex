\chapter{Literature Review}

\section{Modern Bayesian Sampling Techniques in Personalized Medicine}

While techniques to approximate samples from a Bayesian model have existed and evolved over time, the most recent and substantive work has occurred within the last 11 years.  The use of Hamiltonian Monte Carlo (HMC) in applied statistics was initially recognized by Neal in the 1990s \cite{Neal1996-vn}, but entered the main stream in 2011 \cite{neal2011mcmc}.  In 2012, Stan (an open source C++ program to perform Bayesian inference) released their 1.0 version, implementing an adaptive variant of HMC \cite{neal2011mcmc} as well a the no-U-turn sampler \cite{hoffman2014no} which eliminated the need for the user to specify the number of steps the sampler would take in its random walk.  The release of Stan 1.0 resulted in open source tools for performing the most cutting edge techniques for Bayesian inference.  However, it is only recently that HMC's success has been theoretically understood.  In 2014, Betancourt et. al published \textit{The Geometric Foundations of Hamiltonian Monte Carlo} \cite{betancourt2017geometric} which grounded HMC in differential geometry, a field of pure mathematics which statistics practitioners are seldom exposed to.

An important result from this research is that HMC explores the \textit{typical set} of the posterior distribution \cite{Betancourt2017-ak}.  In the typical set, the product of probability density and volume is largest, and hence contributes the most to computations of expectations (which are often expressed as integrals of probability density over volumes in parameter space).  Hence, regions around the mode where density is maximized -- as Maximum A Posteriori does -- may not contribute appreciably to expectations in high dimensional parameter space due to most of the volume being in regions away from the mode than in a neighbourhood around it.  The general conclusion from this research is that HMC is to be preferred to other forms of inference which seek to optimize probability density. 

Despite these findings, Maximum A Posteriori has remained a popular method for performing Bayesian inference, and has seen continued use in the pharmacokinetic and personalized medicine literature as recently as 2020. Brooks et. al \cite{Brooks2016-li} published a review which identified 14 population pharmacokinetic studies which assessed predictive performance of MAP Bayesian estimates of area under the curve (AUC) for tacrolimus  \cite{Brooks2016-li}.  Nguyen et. al used MAP estimates from a Bayesian model to derive phenotyping indexes for use in limited sampling strategies, noting the approach could reduce the time patients spend in hospital waiting to be phenotyped \cite{Nguyen2016-pg}. Preijers et. al use MAP estimation to calculate  pharmacokinetic parameters for a patient undergoing total knee replacement surgery, and use those pharmacokientic parameters to determine a dose required to obtained prescribed factor VIII target levels \cite{Preijers2019-k}.  Finally, Stifft et. al compare predictive performance of a linear regression model with a MAP estimates for a population pharmacokinetic model on tacrolimus trough levels \cite{Stifft2020-uq}.

A one compartment pharmacokinetic model can have three or four pharmacokientic parameters, each of which could potentially have an associated random effect.  The dimensionality of the resulting parameter space for these models can grow very quickly even with a modest number of patients, making differences between MAP and HMC salient.  While MAP can be a good approximation to the posterior distribution for some models, it remains to be seen to what extent MAP offers an appropriate approximation to the posterior of population pharmacokinetic models, in both a predictive and decision making context.

