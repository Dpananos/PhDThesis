\chapter{Paper 2}

\section{Introduction}

% Define personalized medicine, highlight progress
Personalized medicine has been characterized by four goals: 1) to identify drugs for which between-subject variability in effectiveness or toxicity is a key issue for effective treatment, 2) to identify predictors which may explain this variability, 3) to decide on the right dose of the right drug by considering these factors, and 4) to prevent adverse reactions to drugs \cite{morse2015personalized}.  Progress in all four goals has accelerated within the last decade: For example, recent studies on DPYD genotype testing prior to starting fluoropyrimidine-based chemotherapy showed promise in preventing adverse events, making good arguments for integration of DPYD genotype testing into standard of care practices \cite{wigle2019prospective}.  

% Describe static and dynamic personalization.
% Might be nice to have a figure/schematic to show what we mean by static and dynamic personalization. Maybe check SMDM for figure limits.
With regard to the third goal---personalized dosing---the intent of most efforts has been what we call \textit{static} personalization. Such approaches inform dose at one point in time (usually induction) with the goal of eliminating the need for ``trial-and-error'' adjustments (titration) where the dose is adapted to the patient over time in response to its effects, both therapeutic and adverse \cite{morse2015personalized}. Although significant progress has been made, for example in warfarin dosing \cite{gong2011prospective}, the need for titration has been reduced but not eliminated. Thus, there is an opportunity to personalize not only the initial doses but also the titration process to achieve the best result---we call this \textit{dynamic} personalization. This approach has been used in other contexts by applying techniques from disciplines such as control theory, operations research, machine learning, and biostatistics to define and apply models for optimal sequential decision-making for patient care \cite{zhang2021identifying, engelhardt2021importance}.

% Idea: dynamic personalization may be good, but definitely imposes burden. Our framework.
Despite its potential to improve care, dynamic personalization imposes additional burden on the patient and provider, because it requires ongoing monitoring, for example by gathering lab results and returning for additional clinic visits. It is therefore natural to ask whether dynamic personalization is ``worth it.'' Is the additional control over dose worth the additional burden? To help answer this question, we present a unified framework for the development and simulation-based evaluation of static and dynamic personalization based on pharmacokientic (PK) modelling. The knowledge created by our framework can be integrated into a system-level decision-making framework like Know4Go, for example, which can be used to evaluate whether such a personalized medicine program should be implemented into a particular health care system \cite{Martin2016}. Having established our framework, we investigate the static and dynamic personalization of apixaban dosing as a case study.

%Paper structure
We begin in Section~\ref{ss:background} with an overview of dynamic treatment regimes, which underpin our models for dynamic personalization, and we review Bayesian PK modelling, which allow us to predict drug concentrations and to generate simulated patient data. In Section~\ref{ss:framework}, we present our framework, which describes how to estimate  optimal dynamic treatment regimes for personalization by combining Bayesian PK modelling with Q-learning, and describes a simulation-based approach for assessing the potential benefits of different modes of static and dynamic personalization.  We then present our case study of personalized apixaban dosing in Section~\ref{ss:casestudy}. Finally in Section~\ref{ss:discussion} we discuss the results of the case study, and we identify broader issues relevant to the further development and implementation of PK-driven static and dynamic personalization.
