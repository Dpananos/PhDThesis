\chapter{Pooling Pharmacokinetic Information Using Hierarchical Models}


This chapter represents joint work with Drs.\ Simon Bonner and Dan Lizotte.  I am the primary author of this work and was responsible for model design, implementation, as well as writing the manuscript.  Drs.\ Bonner and Lizotte participated in conception and planning, and interpretation of research as well as critical review of the drafted materials.

The motivation behind this chapter comes from the observation that while our clinical pharmacology partners had at least two datasets on apixaban (both used in this study), only one was used to study the effects of novel predictors on apixaban concentrations.  Additionally, my earlier contributions towards studying effects of predictors on apixaban pharmacokinetics relied on the use of linear models.  As is described in the sections that follow, the use of that approach is acceptable under certain circumstances but could be further improved by modelling the pharmacokinetics directly. 

\section{Introduction}

One goal of personalized medicine is optimized dosing of drugs for individuals \cite{morse2015personalized}.  When considering optimal doses, a thorough understanding pharmacokinetic (what the body does to the drug) and/or pharmacodynamic (what the drug does to the body) effects are crucial.  To this end, models describing the mediation of pharmacokinetic/pharmacodynamic effects via clinical, genetic, and lifestyle factors have an important role in deciding which patients should get what dose. Models of this nature are sometimes published by research teams collaborating with drug manufacturers using data from clinical trials.

Independent investigators can find themselves in a situation in which data collection from a particular population of interest is achievable. If the data come from practice (e.g. a personalized medical clinic), there may be questions about how new variables not previously studied in clinical trials affect the pharmacokinetics/pharamcodynamics of a particular drug.  Running large studies in order to examine the effects of these new variables, or discover effects of other variables, may be unrealistic due to a variety of constraints.  Consequently, investigators must think about how best to model the pharmacokinetics, for use in decision making \textit{and} exploration, using the data available to them.

The oral anti-coagulant apixaban provides an illustrative example.  Pharmacokinetic models have been previously published \cite{cirincione2018population,ueshima2018population} in collaboration with the drug's manufacturer using data from clinical trials.  These studies identified age, sex, body weight, renal function,  patient race, and CYP3A4 inhibitors as modulators of apixaban pharmacokinetics \cite{cirincione2018population}, though according to authors the effects of some of these variables were not large enough to require clinical dose adjustment. However, even after adjusting for the aforementioned factors, concentrations of apixaban in real life applications have been observed to be larger than what was reported in clinical trials \cite{sukumar2019apixaban}, raising questions as to the optimal dosing of apixaban for patients in different settings. Additionally, recent research has indicated appropriateness of dose adjustment criteria are unclear \cite{vu2021critical}, citing there is no reduction in safety associated with an increased exposure to apixaban in patients above 75 years of age, below 60kg of body weight, and eGFR (the rate at which the kidneys filter the blood) lower than 50 mL/min. The uncertainty regarding dosing criteria and additional variability in concentrations in day to day use suggests that, while previously published models may be internally valid, these models may not be representative of all populations in which apixaban is to be applied.  That is to say, the models may lack a degree of external validity, thus supporting the idea that pharmacokinetic models may need to be tailored for specific populations of interest. When viewed through a Bayesian lens, the previous modeling work can act as an informative prior on various pharmacokinetic/pharmacodynamic measures.  Creating new models for populations of interest is then more of a "fine tuning" than an all together new approach.  Pharmacokinetic models for use in a specific population may then have two goals:  to adjust doses for a specific population, and/or to explore how additional variables (for example, concomitant medications) not included in the previous studies affect particular parts of the pharmacokinetics of apixaban.  

This study seeks to demonstrate how investigators can fit similar models to their pharmacokinetic data with the aim of accomplishing the goals of accurate modeling of pharmacokientics and exploration of effects of new variables.  We use apixaban as a specific example, but our methodology can be generalized to other drugs for which pharmacokientics are of interest.  Importantly, we only focus on pharmacokientics since blood plasma levels correlate closely with the pharmacodynamic effect of apixaban \cite{byon2019apixaban, upreti2013effect, frost2013safety,frost2013apixaban}. Our approach leverages a Bayesian methodology to building pharmacokinetic models so that we may incorporate prior information from previous studies. Additionally, we describe how investigators can use \textit{all} relevant data available to them to fit these models and make inferences, even if the data are not from controlled studies.  Also, we show how sparsity inducing priors can be applied to new variables in order to explore how those variables may effect apixaban pharmacokinetics, encouraging negligible effect sizes but allowing for large effects to be detected at the cost of a small amount of bias. We present a small simulation study to demonstrate how smallest meaningful effects can be detected through these priors as a function of sample size. Finally, we use an open source Bayesian language to develop our models, making our code freely available.  Previous models are constructed in a proprietary software tool set, which can present a barrier for some. Creation of these models in a free tool removes a barrier to research, making these methods more widely available.
