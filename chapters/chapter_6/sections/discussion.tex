\section{Discussion}


The Bayesian model we present pools information across studies which may differ in study protocol.  Doing so allows investigators to make use of all available data -- be they from controlled studies, or arising from patient interactions in a personalized medicine clinic -- to fine tune pharmacokinetic models to populations of interest. However, this is just one possible model out of a family of similar models. One extension worth mentioning is estimating heterogeneity of effects between studies.  Our model assumes that for a given covariate, the effect is the same in different studies; the effect of weight on the clearance rate is the same across studies, for example.  This need not be assumed, and it may be the case that allowing for heterogeneity of effects between study populations may help explain additional variation beyond what measured covariates already explain.  The extension to include heterogeneity of effect is straightforward for our model, and would see an extended hierarchy considered where each $\beta$ and $\mu$ are generated from some further distribution with unknown parameters. We do not to implement this extension because our data consists of only two studies, making inference on the between study variability in effects difficult to estimate reliably.

To demonstrate how our model can be used to discover novel predictors of pharmacokientics, we included a simulation study in which we place a double exponential prior on a potentially novel covariate's effect on the bioavalability of the drug.  The double exponential prior acts as a sparsity inducing prior, pulling large effects towards 0 as the LASSO does. Our simulation study showed that our model is able to estimate the effect of this novel covariate reliably, even in circumstances where only a small amount of data on repeatedly sampled patients are available to investigators.  The estimates are biased towards the null due to the sparsity inducing prior. This bias attenuates with more data becoming available, and can also change depending on prior hyperparameters. From an estimation perspective, although the estimates of the effects are biased, they may be better suited to predict population effects due to this decrease in variability, similar to the phenomenon displayed by the James-Stein estimator \cite{stein1956inadmissibility,james1992estimation}.  We believe that since the primary goal of exploration is not to get very precise estimates, but to rather to discover new avenues for future research, the exchange of variance for bias is not only acceptable but also preferable.

Finally, we applied our model in a case study of apixaban.  We pooled data from two sources; one from a well controlled clinical study, and the other from an observational setting.  We used a sparsity inducing prior to regularize estimates of the effect of concomitant amiodarone on bioavailability of apixaban.  Concomitant amiodarone's effect on apixaban concentration has been previously studied \cite{gulilat2020drug}, however that model is more descriptive whereas our model is mechanistic (in so far as we model the pharmacokientics explicitly) and incorporates prior information from previous studies.  The findings in our case study and previous work agree; concomitant amiodarone is associated with an increase in apixaban concentrations. It is difficult to evaluate if the magnitudes are similar, however, mainly because our model posits a multiplicative effect whereas previous models assume an additive effect.  However, our model is capable of providing richer inferences due to the mechanistic model and fully Bayesian analysis.  We can propagate uncertainty in the effect of concomitant amiodarone through to other salient pharmacokinetic measures, like max concentration (see figure \ref{fig:max-concentration}).  Additionally, uncertainty in other pharmacokinetic measures can be propagated. For example, where as previous models relied on a point estimate of time to max concentration -- which was the same for each subject -- our model can estimate each patient's time to max concentration (conditioned on covariates) and uncertainty in that estimate propagates through to max concentration.  The posterior distribution of max concentration then captures all uncertainty relevant to the decision, meaning credible intervals should -- at least in theory -- also have better coverage for individuals.  Though a similar Bayesian pharmacokinetic model could be fit using only the sparsely sampled data, pooling information using repeatedly sampled data should be beneficial because of the high precision in estimates of covariate effects afforded by the repeated sampling.

The marginal posterior distributions of the effects display behavior consistent with partially pooled models.  Models fit to each dataset could be considered as a non-pooled estimate, our model -- which combines information from multiple datasets -- can be considered as a pooled estimate.  Partially pooled models have the effect of regularizing estimates towards the population mean, but the size of this regularization depends on the precision and magnitude of the estimate. This behavior is most clearly shown in the radon example provided in chapter 12 of Gelman and Hill's book on multilevel modelling \cite{gelman2006data}.  Gelman and Hill describe an analysis of exposure to radon -- a known carcinogen in high concentrations -- across 3000 counties in the United States.  Within each county, a variable number of houses were measured for radon exposure.  Importantly, some counties had more observations than others.  Within each county, an average level of radon exposure can be estimated.  However, those counties with fewer observations have smaller precision in the estimate.  Those counties with large effects and high precision see little regularization in the estimate of average radon exposure when comparing non-pooled estimates to pooled estimates.  Those counties with large effects and small precision see a strong amount of regularization (see figure 12.1 in  Gelman and Hill \cite{gelman2006data}). Similar explanations can be applied to our model.  As an example, we see that the effect of weight on clearance ($Cl$ in our model) has been regularized to be a compromise of the estimates obtained from models fit on the repeatedly and sparsely sampled data separately.  Additionally, we can see that there is little regularization in effects where one dataset provides a high precision estimate (as in creatinine's effect on clearance). The tendency for partially pooled models to regularize towards the population mean has the effect of trading variance for bias, a theme that has permeated this work.  This should in principle result in estimates of the effects with smaller root mean squared error.

Our study is not without limitations.  Firstly, our repeatedly sampled data come from a study concerning patients with Non-Alcoholic Fatty Liver Disease (NAFLD).  Some patients in this study had NAFLD, others did not.  Our sparsely sampled data did not collect this variable, and so technically it should be considered missing.  One strategy is to impute this variable, be it though frequentist methods or Bayesian methods.  We choose not to impute this and simply exclude it from our model since the study which generated the repeatedly sampled data failed to detect a statistically significant effect of NAFLD on apixaban pharmacokinetics \cite{tirona2018apixaban}. Additionally, because patients in the sparsely sampled data were sampled only once, we were forced to assume their time delay was 0.  The time delay is most probably non-zero, and making this assumption may increase the variability of estimates of other pharmacokientic measures.  Finally, the subjects in each study are quite different with subjects in the repeatedly sampled data being younger, healthier, and with better kidney function on average. Since subjects in the sparsely sampled data are generally different, a linear effect of covariates on pharmacokientic parameters may not be appropriate due to extrapolation.  A possible remedy for this would be to model the effects of covariates using splines or other suitably flexible methods.  If additional subject matter expertise is available, investigators may choose to model the effects with monotone-splines.  We believe specifications about the functional form of effects are best done with the aid of subject matter experts (pharmacologists, physicians, etc) and opt for the simplest non-trivial functional form for our effects.

\section{Conclusion}


In this study, we demonstrated how investigators could accomplish the goals of accurate modelling of pharmacokientics and exploration of new variables via the use of a heirarchical Bayesian model of pharmacokientics. The model pools information from multiple studies and shrinks estimates of effects.  The result is a trade off of variance for bias, which should improve predictive accuracy.  Additioanlly, we peformed a simulation study to demonstrate how sparisty inducing priors can be used to indentify effects of new variables.  We showed that, even in small samples, a small amount of bias is observed in estimates of novel effects and this bias attenuates with more data. Future research may include modelling heterogeneity of effect by adding another level to the hierarchy.